\documentclass{article}

% Language setting
% Replace `english' with e.g. `spanish' to change the document language
\usepackage[english]{babel}

% Set page size and margins
% Replace `letterpaper' with `a4paper' for UK/EU standard size
\usepackage[letterpaper,top=2cm,bottom=2cm,left=3cm,right=3cm,marginparwidth=1.75cm]{geometry}

\usepackage{blindtext}
\usepackage{multicol}



% Useful packages
\usepackage{amsmath}
\usepackage{graphicx}
\usepackage[colorlinks=true, allcolors=blue]{hyperref}

\title{School: A Crowd-Based Capital Allocation System}
\author{David Hopkins \\ th340d.co}
\date{}

\begin{document}
\maketitle

\begin{multicols}{2}



\begin{abstract}
While there are evident economic and social benefits to combining both crowdfunding and crowdsourcing activities into a single online marketplace, to date these networks have largely been treated as independent. This paper considers whether a dearth of appropriate tools has been preventing these two forms of activity from being brought together, and in particular considers the utility of diagrams as a new means of organize and directing capital and labor. After exploring the nature and function of diagrams as a planning tool, we find that they can be used effectively for this purpose and two novel types of diagram are introduced explicitly to that end: Directive Trees and DQC (Directive-Question-Command) Diagrams. Equipped with these tools, we proceed to outline a comprehensive technical solution that combines crowdsourcing and crowdfunding, which we entitle School. We briefly discuss matters pertinent to the implementation of the School system and conclude with thoughts about the future of crowdfunding and crowdsourcing technologies.

\end{abstract}

\section{Introduction}

Crowdfunding and crowdsourcing are both tactics that have utilised social networking technology to allow economic actors to achieve outcomes more quickly. Platforms catering to each of these two strategies match donors of resources (money in the case of crowdfunding and labor in the case of crowdsourcing) with those who are able to provide them with value in return. This is typically achieved by (a) pulling together as wide and deep a pool of users as possible and (b) making listings as easy as possible to create, browse and engage with. As these two forms of technology use similar dynamics, one naturally wonders whether it is possible for them to be combined together into a single platform. If a system such as this was realizable, its advantages would be obvious: it would offer a considerable productivity boost to its userbase by improving the momentum with which value moves around the economy. It is likely that any community with such a tool available would be able to pursue a wider and deeper range of economic activities than otherwise and that, in turn, this greater level of flexibility would subsequently lead to the employment of a considerable amount of otherwise latent talent and capture returns on capital that would otherwise be unrealized.

It is also worth mentioning that the ability of a community to bring together its efforts for petitioning for and deploying capital is likely related to its capacity of self-sufficiency. A single, open system makes it not only easier for individuals to manage their own personal lives but for them to get a bird's-eye view of the interests of the broader community, improving public awareness of civic issues and increasing the sophistication with which they are considered.

In such ways, the platform may begin to take on "a life of its own."

Given that the synergies seem obvious, why is it that such a solution has not been implemented already, particularly given that crowdfunding and crowdsourcing mechanics are now reasonably well-understood and offered within competitive industries? We propose that there has been a legitimate limitation impeding the arisal of a general, crowd-based capital allocation platform in the form of a scarcity of efficient methods to integrate crowdfunding and crowdsourcing listings. Without a simple and unified approach to representing each of these two forms of data internally, the cost for platforms in catering to both markets becomes prohibitive.

However, recently a possible solution to the problem of data representation was posited by the organizers of the Mapsmap challenge. In particular, Protocol Labs posed an open question to its audience as to whether diagrams could be used to coordinate crowdfunding and crowdsourcing activities and as the basis for an app that combines the two markets \phantom{[1]}.

Given their lateral nature, it can be predicted with near-certainty that diagrams are a class of data representation tool that early 21st century society is under-utilizing \phantom{[2]}. Diagrams assist people to find links between seemingly disparate entities to spawn new ideas and to identify patterns that would otherwise be obscured. These qualities make diagrams a highly-effective tool for visioning and innovation - and for "striving for the exceptional" rather than "improving the average."

Protocol Labs' proposition induces us to consider whether the power of diagrams (in revealing possibilities for future exploration that have previously gone unconsidered) could act as a kind of magnetic force that pulls us through whatever barriers have prevented crowdsourcing and crowdfunding from being united in the past - and allow people to have a greater level of economic autonomy than ever before.

In this paper, we provide an answer to that question. More exactly, we confirm that the original proposition is sound. However, we also go much further, and explain exactly how it may be realized technologically.

To do this, we address the elements of the problem of implementation that, as it turns out, are counter-intuitive, and these are disassembled and clarified, observe the correct solution naturally revealing itself.

\section{Challenges}

\subsection{Relating to the utility of diagrams}

The first presupposition we shall address is the idea that a useful software tool can be constructed in which a large diagram is worked on remotely by a crowd of people that is in turn useful for a crowd of people to use as a tool to coordinate work.

We refute this presumption by establishing the following principle:

\begin{quote}

\textit{Principle 1. The decipherability of a diagram is correlated inversely and exponentially to the resolution of the data which it attempts to depict.
}
\end{quote} \phantom{[FIGURE 1]}

By way of illustration, a line chart with a single line on it is exponentially more readable than a chart with ten lines on it. By the time that the line chart reaches one hundred lines, it has already lost almost all of its utility. \phantom{[FIGURE 2]}


Similarly, a two-dimensional star chart is exponentially easier to intuit conclusions from than a three-dimensional one, while a four-dimensional one is practically unintelligible.

One finds that this dynamic continues to exist regardless of the type of diagram that one is considering.

\subsection{Relating to the functionality of diagrams}

The second presupposition to be addressed is that it is useful to represent data at the following two scales on a single diagram: (a) classes of new technologies (such as has been represented on tech trees historically) and (b) units of work that are suitable for the typical members of a crowdsourcing app's worker population.

We dispose of this idea quickly by calculating the difference between a new class of technology and a unit of work that can be comfortably handled by an individual. We may do this by taking a representative sample of each, such as (a) the notion of "scientific calculators (including the invention, adoption and maturity product of the product and industry sufficiently for future technologies to be built on top of it)" and (b) "fixing a roof tile" and find the scale difference to be roughly in the order of 10^9. \phantom{(XX)}

It is untenable to represent data at every scale in between these two extremes on a single diagram at sufficient resolution to be useful for the user. (While it is not infeasible that a visual tool could be used to represent this quantity of data, such a tool would likely take on the character of a virtual world \phantom{[3]}.)

\subsection{Relating to the nature of progress}

In addition to the two previous refutations being arguments against the utility of universal, central diagrams, and even more strongly so when considered in combination, there is a fundamental point to be made with relation to how technological progress occurs. In particular, that the two following assumptions may be considered traps:

\begin{quote}

(1) that the majority of the work required relating to any technological innovation for civilization to be able to progress beyond it is related to the conception of the invention, and, relatedly:

(2) that technological innovation is haphazard and entirely unpredictable, with there being multiple possible sequences whereby technology may be adopted.
    
\end{quote}


While the single most valuable act involved in the development of a new technology may be the moment of inspiration at which it is invented, which perhaps concludes at the moment when a single person becomes convinced that the technology works to address a particular problem while being able to articulate every element of the solution required for it to work (or in sufficient detail that a layperson could intuit the rest, at least), this is naturally only the beginning of a technology's lifecycle.  The tendency for inventors to over-estimate the value of having a good idea and under-estimate the effort involved in finding a market for that idea has been well-documented.

During the course of its journey to market, the product goes through many iterations and an entire infrastructure is built around it to be enable to be produced. A good technology will then have an entire industry arise as an ecosystem which allows it to be developed further, while all the processes surrounding it are refined. Even for small innovations, it is only once the technology has reached a certain level of penetration and a certain level of sophistication from its necessarily crude beginnings that others can take advantage of it and when innovating upon it, it is likely to be an entire ecosystem, including the marketers, manufacturers, regulators and consumers, that one is ultimately basing an innovation upon.

If we present the steelman of the alternative view, by asking ourselves whether this same dynamic applies for the innovation of ideas or for software, both of which are multi-faceted and can build upon themselves without seemingly needing much contact with the rest of the world, we find that this falls apart by addressing the second of the two points above.

We find that, entirely unlike a circumstance in which one knows the answer at the outset, the act of creativity (which includes any projection of how one wishes to manoeuvre the world towards a better future) necessitates that actions be built upon prior actions. In particular, every act of human creation is, by necessity, an act of combining pre-existing things in a novel way. The further into the future one projects activities, the more things that don't exist already are combined together and so the more context is required for any diagram which would seek to communicate it (and diagrams can't handle complexity).

Instead, to discuss the future in any meaningful way (which ultimately means to actually be talking about the present, as the future does not have any meaning), one must refer to trends, patterns, directions and outcomes. For example, it is perfectly fine to say, "I will go for a stroll around my neighbourhood to relax later this afternoon," and yet it requires an unprecedented amount of planning to say with confidence, "I will go for a stroll around my neighbourhood to relax 8,000 hours from now," if there were any point in saying it at all.

Instead, one refers to cycles and trends. One may say, "Every year, on my wedding anniversary, I commit to buying my spouse a gift that revitalizes and deepens our love for one another," or, "I will walk for an hour every day until I have found my ideal weight."

Cycles and trends, however, are not particularly well expressed by knowledge diagrams such as trees, unless their explicit purpose is to demonstrate a particular cycle or trend and they are sliced down accordingly.

We also see that the actual evolution of technologies, and particularly the order in which they become a part of the fabric of society, obeys patterns with complete reliability. In other words, while it may be possible for a diagram to be used to plan out further and further into the future (provided that it is done by an individual and not a crowd, so that context can be foregone), looking further into the future is all that the diagram achieves (and indeed one can imagine the possibility of getting lost in it). As soon as one attempts to implement the contents of the diagram in any real fashion, one quickly finds that no-one else understands a word of it. One must go back and execute the mundane and comparatively gruelling work of actually implementing each technology and meanwhile the diagram gets put in the drawer.

There is no way around putting the work in to every technology and there is also no way of communicating what work is to be done to implement any given technology far into the future. One relies on diagrams or any other form of communication to describe things in the cleanest and broadest terms, which is in terms of principles and objectives, and not in terms of actions.

Finally, as to the point of technologies always being successfully adopted in unaltering patterns (perhaps regardless of their time of invention), one may find this substantiated in the history every of possible arena, including the silhouettes of women's dresses (which changes from pencil-like, to bustle-like to hoop-like and back to pencil-like) to communications technology (which changes from being centralizing to decentralizing and back to centralizing) to software products (which bundle and then unbundle and then bundle again).

\section{Implementation}

\begin{quote}

\textit{\phantom
{Principle 2. Everything that can be represented as a diagram can be represented using a list and vice versa.}
}
\end{quote}


We may consider an effective solution as hinging on answering the following two questions:

\begin{quote}
(1) Can both crowdfunding and crowdsourcing listings be represented using a single framework that is both cohesive and scaleable?

(2) Can the use of diagrams engage and incent users to contribute to the network?

\end{quote}
Framed differently, our hypothetical technology must be both (1) \emph{innovative} and (2) \emph{marketable}.

Because diagrams and lists are always interchangeable, it is not important to consider at this stage which tool is used to represent the data but rather how it is structured.

Particularly, in order to scale the use of our system, it will be necessary for data to be stored as homogenously as possible. For example, it is not efficient to store pricing data in a database variously as numbers and as strings and there is very little that one can do with the data until a consistent format has been settled upon and all entries have been converted into that format.

Compared to something like pricing data, job descriptions are woefully inefficient. In part, job briefings have had a dual function of acting like an advert, although the improved speed of creating and executing tasks likely makes this element superfluous and we can instead focus specifically on how to communicate what is to be done most elegantly, while allowing monetary rewards to be easily understood and compared and for the money to "speak for itself." More than that, if workers have considerable freedom to complete and move between tasks, it is feasible that they are able to create their own thesis or mission surrounding the tasks that they choose, provided they are given suitable tools to find and organise work (and we will return to this idea later).

As we are interested in how information is communicated, we will also see that language is of great importance to our deconstruction.


Suppose that what we are looking for is an elegant way to facilitate the communication of ideas and solutions amongst a community of people, in which money is continually transfered from issuers of ideas to issuers of solutions, with people being able to move back and forth between issuing ideas or solutions at any time.

The fundamental unit at the heart of this (or any!) economy may be expressed as follows:

Problem > Solution

Or alternatively,

Wish > Implementation

For the time being, we will treat "idea" "problem", "proposal for a solution to a problem" and "wish" as the same, as we are currently most interested in the transition to a result/ implementation/desired outcome/solution.

While there are different forms of financing this process, we will use what is perhaps the simplest, namely bounty-based model. In this model, a person issues the idea that they wish to have implemented and when a person delivers the desired result, they are rewarded by the issuing party. A successful result can be submitted at any time and the offer stands until the prize is paid or the issuer rescinds it (with some system of safeguards in place to prevent the issuer from not accepting a legitimately prizeworthy attempt or otherwise stealing the result of the participant's labor).

We find that the "Idea" can be deconstructed concisely into the following elements, in order:

Name > Directive > Question > Command

The Name is the wish described as a title. This element is optional if the other elements are included. It may be expressed using a proper noun (titlecase) noun and is similar to how a toddler or oafish nuclear power plant worker might express what they want (i.e. "Mm... Doughnut!"). A Name may optionally be succeeded by an exclamation mark.

The Directive is the Name described actively, in terms of what the wisher wants to do. This is perhaps the most important of the four elements as it is the least replaceable. While reasoning must be applied to turn an abstract concept into a directive, doing so is usually trivial. A directive may be expressed using all uppercase, without any end-of-sentence punctuation and begins with a verb.

E.g. "EAT A DOUGHNUT"

It is important to identify that it is not actually required yet to have any real contextual understanding of what is being expressed. What makes something a Directive is that it is intrinsically motivating. Unlike a Command, which requires a reason for doing, the Directive is distinguished by the fact that it acts as its own reward. At this stage of a wish or idea, it may be said that we simply want what we want.

A Directive establishes that we have a desire to do something, whereas a Name simply establishes that a thing exists, and so often we will identity a wish when it is a directive. For example, as a verbal completion of the phrase, "I want to..."

We express the directive without punctuation to help to represent how it is naturally understood by us, which is, like Names, in the form of sound. A Directive is not actually a sentence as even punctuation forms an unnecessary level of context at this stage.

In other words, a baby may understand the phrases "give me" and "biscuit" and can use this alone to sufficiently both understand and communicate what it wants, prior to being able to convert this into a concept that can be meaningful independent of context (a sentence).

A Question is phrased as one might expect - starting with a question word and ending with a question mark - and formulating it is the most lateral of the steps. Converting a Directive into a Question involves identifying what it is about the Directive that is unknown.

Converting a Directive into a Question carries an implicit suggestion, although quite a logical one, that the distance between wanting to do something and doing that which one wants to be doing is ultimately a difference in perspective, or at least more of a difference in perspective than it is of anything else. Whether this principle is true or not, and whether this process bears a relationship to how a person naturally undertakes to get the things that he or she wants if it were to be articulated is less important for present purposes than the result that it yields. While in modern times it might be considered impractically idealistic (albeit true in the abstract), in a matter of language such as this we find that it offers a necessary level of precision that is otherwise unavailable to us.

To turn a Directive into a Question, a person interrogates the Directive with a sentiment akin to, "If I know that I want to be doing this, than why aren't I?"

If such a question can be phrased non-judgmentally and to the affirmative, it becomes something like, "What, if I knew, would lead me to be doing this thing that I want to do right now?"

By asking this lateral question, the person for the first time in this process gives acknowledgment to the possibility that what is wanted is achievable and now has laid the foundation for a path to achieve it, as, necessarily, any true answer to the question involves the person the doing it (or else, it wouldn't be a true answer).

The rest of the process of converting the idea into reality then becomes the almost perfunctory task of describing whatever actions may be taken that would feasibly offer the possibility of answering the posited question.

Alternatively, if no actions can be found, one can simply refine the question. This would be akin to asking something in the nature of, "What would (or even might) be the first step to finding what the first step is?"

Ultimately, one can become so specific with the question that is being posed that a failure to come up with an answer could only be accompanied by a total logjam.

\end{multicols}
\end{document}
